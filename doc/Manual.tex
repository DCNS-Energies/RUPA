%Préambule du document :
\documentclass[11pt]{report}
\usepackage[latin1]{inputenc}
\usepackage[T1]{fontenc} 
\usepackage[francais]{babel}
%\usepackage{multirow}
\usepackage{graphicx}
\usepackage{hhline}
\usepackage{hyperref}
\usepackage{titlesec}
\usepackage{tabularx}
\usepackage{amsmath}
\usepackage{amssymb}
\usepackage{pdfpages}
\usepackage{array}


\titleformat{\chapter}{\normalfont\huge}{}{20pt}{\huge \bfseries}
\hypersetup{
    colorlinks,
    citecolor=black,
    filecolor=black,
    linkcolor=black,
    urlcolor=blue
}

%\renewcommand{\chaptername}{}
\renewcommand{\thechapter}{}
\renewcommand{\thesection}{}
\renewcommand{\thesubsection}{}

\title{Recover Underwater Position by Acoustic \\ RUPA}
\author{Tristan Dewalle \\ Intern in DCNS Energies}
\date{From 06/03/2017\\To 20/07/2017}

%Corps du document :
\begin{document}
    \maketitle
	\tableofcontents
	\chapter{Introduduction}
	    %\section{Really short description of DCNS Energies}
	%	DCNS Energies is a french firm, subsidiary of DCNS. It is 
	 %   making R&D on Marine Renewable Energie, so sometimes, underwater structures are deployed
	 RUPA is an OpenCPN plugin, designed during my internship in DCNS Energies. The aim of that plugin is to 
	 have a friendly user HMI to locate underwater structures. To make that tool more efficient, I extended it 
	 to the management of the life of campaigns of measurement. In that manual, I will describe you how to use it
	 (quite simple, don't worry) and if someone want to use it to improve it, in the "Developper part" I will 
	 explicit what my logic was.

	\chapter{User part}
	    I hope this tool is quite easy to use, but here are some instructions, just in case.

	    Please, refer to the glossary if you're not sure what I'm talking about.
	    \section{About the devices }
		About software, RUPA was developped with:
		\begin{itemize}
		    \item Linux Mint 18.1 Serena
		    \item OpenCPN 4.4.0
		\end{itemize}

		 About the hardware: 
		\begin{itemize}
		    \item A deck unit from Sonardyne
		    \item A Transponder from Sonardyne
		    \item A FTDI (to connect computer and deck unit via USB)
		    \item A GPS from trimble (just for information)
		\end{itemize}
	    \chapter{Installation}
            You have to install:
            \begin{itemize}
                \item opencpn (not needed for compilation, but as it's a plugin for it...)
                \item gcc-multilib
                \item g++
                \item autotools-dev
                \item cmake
                \item wx3.0-headers
                \item liwxgtk3.0-dev
                \item libghc-zlib-dev
                \item gettext
                \item libbz2-dev
                \item libmysqlcppconn-dev 
                \item mysql-client
                \item mysql-server
            \end{itemize}
        \vspace{3cm}
            You have to install ftd2xx too. To do that:
            \begin{itemize}
                \item go to \url{http://www.ftdichip.com/Drivers/D2XX.htm}
                \item download the good version (actually in the linux row, and for me the third or fourth column)
                \item go where you downloaded it
                \item extract it
                \item open a terminal where the folder is extracted
                \item sudo cp ./releases/build/lib* /usr/local/lib
                \item cd /usr/local/lib
                \item sudo ln -s libftd2xx.so.1.1.12 libftd2xx.so
                \item sudo chmod 0755 libftd2xx.so.1.1.12
            \end{itemize}
            If you want more explanation, please, look at the "FTDI Drivers Installation Guide for Linux".
            %If you cannot download mysqlcppconn, you can:
            %\begin{enumerate}
            %    \item git clone https://github.com/mysql/mysql-connector-cpp.git
            %    \item cd path/to/mysql-connector-cpp
            %    \item cmake .
            %    \item make
            %    \item sudo make install
            %\end{enumerate}
            %From \url{https://dev.mysql.com/doc/connector-cpp/en/connector-cpp-installation-source-unix.html} 

        \vspace{3cm}
            Before launching it, you have to use the database:
            \begin{itemize}
                \item mysql -u root -p
                \item type the password you entered during installation
                \item GRANT ALL PRIVILEGES ON rupa.* TO 'youruser'@'localhost' IDENTIFIED BY 'password' 
                \item SET PASSWORD FOR 'youruser'@'localhost' = PASSWORD('')(this step is deleting the password but if you want to keep one, skip that step and set it in RUPA/src/RUPA\_Utility.h, l.62)
                \item quit
                \item mysql
                \item create database rupa
                \item quit
            \end{itemize}

        \vspace{3cm}
            Then build RUPA:
            \begin{itemize}
                \item cd /path/where/you/want/to/download/rupa
                \item git clone https://github.com/DCNS-Energies/RUPA.git (if not already done)
                \item cd RUPA/build
                \item rm -rf * (to correct something that have to be fix no need for sudo)
                \item cmake .. .
                \item make
                \item sudo cp librupas\_pi.so /usr/lib/opencpn
                \item cd ../database
                \item mysql rupa<schema.sql
            \end {itemize}
            
	    \chapter{Usage}
        \section{Quick guide}
        Supposing you've already instlled rupa, you will basically do that:
        \begin{enumerate}

            \item Enable RUPA in OpenCPN 
            \item click on the RUPA logo (even if actually it's the dog from watchdog)
            \item click on "New Campaign"
            \item fill the fields
            \item click on "Install Now"
            \item click on "New Structure"
            \item fill the fields
            \item click on "Add Transponder"
            \item fill the fields
            \item redo the 8th step as many time as needed to list each transponder on the structure you want to locate
            \item click on "Launch Semi-Auto Burst" in the "Manage Structure" window
            \item click on the "Range" button as many time as wanted
            \item redo the previous step as many time as wanted
        \end{enumerate}
        \vspace{1cm}
        If you want to recover a Structure:
        \begin{enumerate}
            \item click on the RUPA logo (even if actually it's the dog from watchdog)
            \item select the Campaign where the structure is and click on "manage" button
            \item select the Structure you want to recover, click on manage
            \item select the "Recovery" tab
            \item launch the Semi Auto Burst
        \end{enumerate}
        \vspace{1cm}
        \begin{itemize}
            \item if you want to mark a Campaign or a Structure as recovered, select it and click on the (Un)Revover button
            \item if you try to delete something, there will be no confirmation message, and no way to recover it
        \end{itemize}
\newpage
        \section{Detailed guide}
		\subsection{Campaign}
		    It is supposed you have clicked on the "RUPA" button, while you're in OpenCPN.
		    Here, you have 2 tabs and 5 buttons. The "Current Campaign" tab list what campaigns are not finished
		    and "Finished Campaign" list what campaigns are finished.
		    \paragraph{The "New Campaign" button} allow you to enter the minimal information you use find usefull to describe
		    a campaign (the geographical area, the name of th campaign). Be careful, at this time, it's not possible 
		    to edit that simply.
		    \paragraph{The "Delete Campaign" button} remove all the datas you had got: what you entered thank's to the "New 
		    Campaign" button, but every measurement you should have done. Be really carefull, there is no way to 
		    recover what you erased (really, there is absolutely no tricks to do that).
		    \paragraph{The "Manage Campaign" button} open a new window : "Manage Campaign" in wich you can see each structure.
		    \paragraph{The "(Un)Finish Campaign" button} set a Campaign as finished or current, and move it to the corresponding table.
            \paragraph{Make a backup of all datas, recursively} not working yet

        \vspace{1cm}
		\subsection{New Campaign}
		    \paragraph{The "Campaign's Name" field} let you name the campaign as you want.
		    \paragraph{The "Geographical Area" field} let you indicate where the campaign takes place.
		    \paragraph{The "Install Later" button} will save what you typed and show you the "Campaign" window.
		    \paragraph{The "Install Now" button} will save what you types too, but here, it redirect you to the "Manage Campaign" window.

        \vspace{1cm}
		\subsection{Manage Campaign}
            \paragraph{The "New Structure" button} Open the "Manage Structure" and "Set Up Structure" windows. It creates a nex Structure by the way.
            \paragraph{The "Delete Structure" button} Delete the Structure selected and all of its Measurement, Transponder, etc. Be carefull, there is no confirmation window.
            \paragraph{The "Manage Structure" button}Open th "Manage Structure" window for the Structure you selected
            \paragraph{The "(Un)Recovered" button}Set a Structure as current or recovered, and move it to the corresponding table.
            
        \vspace{1cm}
        \subsection{Set up Structure}
            \paragraph{The "Location Name" field}Allow you to give a specific name to the structure
            \paragraph{The "Devices" field} you can list the devices (sensors or other) you put on the structure

        \vspace{1cm}
        \subsection{Manage Structure}
            \paragraph{The "Set Up General Settings Of Structure" button} Open the "Set Up Structure" window for the current Structure.
            \paragraph{The "Add Transponder" button} Open the "Manage Transponder" Window and add a transponder to the list.
            \paragraph{The "Remove Transponder" button}Remove the selected transponder, without confirmation window.
            \paragraph{The "Launch Auto Burst" button} Not working yet.
            \paragraph{The "Launch Semi Auto Burst" button} Open the "Semi Automatic Burst" window wich allow to make distance measurement. You need to add at least a 
            transponder before clickingon that, else, OpenCPN will crash.
            \paragraph{The "Enter a burst manually" button} Not working yet.
            \paragraph{The "Edit Burst" button}Open Burst Editing window.
            \paragraph{The "Delete Burst" button} Delete a burst without asking you to confirm (the confirmation window wich pop is actually useless).

        \vspace{1cm}
        \subsection{Manage Transponder}
            \paragraph{The "Address" and "Frequency" fields} Used by Sonardyne to send message as : address:frequency.
            \paragraph{The "Serial Number" field} Serial number of the transponder, fill it if you want.
            \paragraph{The "Battery Voltage (V)" field} You can fill it (or not) manually or by clicking on "Check Battery Automatically"

        \vspace{1cm}
        \subsection{Semi Automatic Burst}
            \paragraph{The "Message Sent" field}Count how many time you clicked on the "Range" button.
            \paragraph{The "Message Received" field}Count how many time you received a useful answer.
            \paragraph{The "Last value of received message"" field} print the value received by the computer (or an error message).
            \paragraph{The "Range" button} the button to get the distance between the probe and the transponder.
            \paragraph{The "Next Transponder" button} switch to th next transponder, according the table where transponders are listed.
            \paragraph{The "Previous Transponder" button} switch to th next transponder, according the table where transponders are listed.
            \paragraph{The "Finish Burst" button} Close the window, all the Measurement are saved when they are acquired.

        \vspace{1cm}
        \subsection{Burst Editing}
            \paragraph{The "Add Measurement" button}not working yet
            \paragraph{The "Delete Measurement" button}Delete the selected Measurement
            \paragraph{The "Add Date" button}Not working yet.

%	\chapter{Developper part}



\iffalse		
\chapter{chap}
	\section{sect}
		\subsection{sub}
		\begin{itemize}
		\end{itemize}
		
		\begin{enumerate}
		\end{enumerate}
		
					\begin{center} \begin{minipage}{\textwidth}
					\renewcommand{\footnoterule}{}
					\footnotetext[1]{laurem}
					\begin{tabular}{|l|l|l| || r|r|c|}
						\hline
						\multicolumn{6}{|c|}{tab}\\
						\hline
						 \\ \hline
						 \footnotemark[1]\\ \cline{3-3}
						 \multirow{6}{*}{F 1} 
%							 \multicolumn{2}{|c||}{Changement de châssis}& \multicolumn{2}{c|}{Rehausser les pièces pour le pivot}\\

%						\end{tabular}
%						\end{minipage} \end{center}

						
%						\url{perdu.com}\\
					
					
					
				
%			\appendix
%			\chapter{documents annexes, premiere div}
%			\includepdf[pages={1}]{rep/doc}
		
		\fi
\end{document}

